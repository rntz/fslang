%% Adapted and extended from my 2-page HOPE 2025 submission.
%%
%% consider: format = acmsmall / sigplan / acmengage / sigconf / acmtog
%% consider: review = true
\documentclass[sigplan, screen, dvipsnames]{acmart}
\settopmatter{printacmref=false}
\setcopyright{none}
\renewcommand\footnotetextcopyrightpermission[1]{}
%\pagestyle{plain}


%% ---- Top matter ----
\title{Finite functional programming}
%\title{(Finite function)al programming}
%\title{Toward finite functional programming}
\title{Finite functional logic programming}
%\title{Functional programming with support}
%\title{Supporting functional programming} % lol
%\title{Graded, grounded functional logic programming}
%\title{Functional logic programming, graded \& grounded} % nice
%\title{Logic programming with finite functions}
%\title{Adding support to functional programming}
%\title{Logic is functions with support}
%\title{logic = functions + support}
%\title{Logic, functions \& support}
%\title{Functions with support}

\author{Michael Arntzenius}
\affiliation{%
  \institution{UC Berkeley}
  \city{Berkeley}
  \country{USA}
}
\email{daekharel@gmail.com}

\acmConference{Work in progress}{2025}{Berkeley}


%% ---- Packages ----
%\usepackage{anyfontsize} % silence font size not available warnings by autoscaling fonts
\usepackage{amsmath}
%\usepackage{latexsym,stmaryrd}
%\usepackage{mathtools}          % \dblcolon
\usepackage{mathpartir}         % mathpar
\usepackage{nccmath}            % fleqn (for code environment)

\usepackage{minted}
\usepackage{listings}           % lstlisting
\lstset{language=prolog, columns=fullflexible,
  basicstyle=\ttfamily, commentstyle=\color{Green}}

\usepackage{hyperref,url}
\usepackage[nameinlink,noabbrev]{cleveref}
\newcommand\crefrangeconjunction{--} % use en-dashes for ranges.


%% ---------- Code environment ----------
\newlength\codeoffset
\setlength\codeoffset{0pt}

\NewEnviron{code}{
  \begin{fleqn}[\codeoffset]
    \begin{flalign*}
      \setlength\arraycolsep{0pt} % necessary for flush left alignment.
      \begin{array}{l}\BODY\end{array}
  \end{flalign*}
  \end{fleqn}
}


%% ---------- \negphantom command ----------
%% https://tex.stackexchange.com/questions/316426/negative-phantom-inside-equations
\makeatletter
\newlength{\negph@wd}
\DeclareRobustCommand{\negphantom}[1]{%
  \ifmmode
    \mathpalette\negph@math{#1}%
  \else
    \negph@do{#1}%
  \fi
}
\newcommand{\negph@math}[2]{\negph@do{$\m@th#1#2$}}
\newcommand{\negph@do}[1]{%
  \settowidth{\negph@wd}{#1}%
  \hspace*{-\negph@wd}%
}\makeatother


%% ---- Commands ----
\newcommand\todo[1]{{\color{ACMRed}#1}}
\newcommand\den[1]{\llbracket{#1}\rrbracket}

\newcommand\bnfeq{\Coloneqq}
\newcommand\bnfor{\mathrel{\,|\,}}

\newcommand\name[1]{\ensuremath{\mathit{#1}}}
\newcommand\fnspace{\;}
\newcommand\fn[1]{\lambda{#1}.\fnspace}
\newcommand\fa[1]{\forall{#1}.\ }
\newcommand\<\;                 %function application space

\newcommand\cto{\Rightarrow}    % "cartesian \to"
\newcommand\lto\multimap        % "linear \to"
%\newcommand\fto{\underset{\textsf{fin}}{\to}}           % finite map
\newcommand\fto{\rightarrow}    % "finitely supported \to"
\newcommand\x\times
\newcommand\with{\mathbin{\&}}
\newcommand\ox\otimes
\newcommand\tbool{\name{bool}}
\newcommand\tmaybe[1]{\ensuremath{{#1}_{*}}}
\renewcommand\tmaybe[1]{\ensuremath{\name{maybe}~{#1}}}

\newcommand\undertype[2]{\underaccent{\textsf{\upshape\scshape#1}}{#2}}
\newcommand\eqt[1]{\undertype{eq}{#1}}

\newcommand\G\Gamma
\newcommand\D\Delta
\renewcommand\O\Omega
\renewcommand\L\Lambda

%\newcommand\emptycx\varepsilon
\newcommand\emptycx{{\cdot}}
\newcommand\of{\mathbin{:}}
\newcommand\cxsep{\mathbin{/}}
\newcommand\hyp[2]{{#1} \of {#2}}

\newcommand\cJ[3]{{#3} \,\vdash\, {#1} : {#2}}
\newcommand\J[5]{{#3} \cxsep {#4} \cxsep {#5} \,\vdash\, {#1} : {#2}}
%\newcommand\pJ[5]{{#3} \cxsep {#4} \cxsep {#1} : {#2} \,\vdash\, {#5}}

\newcommand\const\name
\newcommand\cname\text          %constructor name
\newcommand\cnil{\cname{nil}}
\newcommand\cjust{\cname{just}}
\newcommand\ctrue{\cname{true}}
\newcommand\cfalse{\cname{false}}

\newcommand\tnil\cnil
\newcommand\tjust[1]{\cjust\<{#1}}
\newcommand\letin[2]{\textbf{let}~{#1}~\textbf{in}~{#2}}
\newcommand\leteq[3]{\letin{{#1} = {#2}}{#3}}
\newcommand\letjust[3]{\leteq{\tjust {#1}}{#2}{#3}}
\newcommand\letpair[4]{\leteq{({#1}, {#2})}{#3}{#4}}
\newcommand\ttrue{\ctrue}
\newcommand\tfalse{\cfalse}
\newcommand\tabsurd{\name{absurd}\<}
\newcommand\tand{\mathrel{\mathbf{and}}}
\newcommand\tor{\mathrel{\mathbf{or}}}


\begin{document}
\begin{abstract}
  I propose unifying functional and logic/relational programming by treating predicates/relations as functions equipped with their support -- the set of inputs whose output is non-false.
  For instance, the logic programming language Datalog, in which all relations are finite, is a language of finitely supported boolean functions.
  %
  Finite support allows representing the function, not as code, but as data -- as a table of input-output pairs -- making bulk operations more efficient.
%
  If we combine finitely supported functions, represented as data, and higher order functions, represented as code, in the same language, we get what I call \emph{finite functional programming.}
  %
  To this end, I demonstrate a simple type system to check finite support by treating variable grounding as a graded effect.
  %Aggregation and other Datalog extensions (such as bag and semiring semantics) are naturally handled by generalizing from boolean functions to arbitrary pointed output types.
  %% \todo{do I actually do this? $\rightarrow$}
  %% I sketch future work on semantics and implementation strategies, particularly for recursion.
\end{abstract}

\maketitle


%% \section{Motivation}

%% How do we type this? By treating variable grounding as an `effect,' of sorts. (or is it a coeffect)

%% integrating relational and functional programming; Datalog in the large; modular and higher-order code.

%% expanding Datalog to key-value maps eg semilattice semantics, semiring semantics.
%% making it more natural to write aggregations etc.

%% propose to think of logic programming as being about, not ``relations'', but \emph{functions}. a relation is really a function into booleans. once we take this point of view, key-value maps are an obvious generalization.

%% what's special about relational programming compared with functional programming is that functions are \emph{not} treated as black-box input-output transformations, but can also be run ``backwards''. in particular, we can query the \emph{support} of a function.

%% \todo{TODO}


\section{Logic programming is (not) functional}

Consider the following simple logic program to find mutual follows in a social network graph:

\begin{lstlisting}
follows(john, mary).
follows(mary, john).
... % many more follows() facts.
mutuals(X,Y) :- follows(X,Y), follows(Y,X).
\end{lstlisting}

\noindent
Predicates like \texttt{follows} and \texttt{mutuals} denote functions into booleans.
Following this intuition, we can translate this logic program directly into a functional one:

\begin{code}
\name{follows},\, \name{mutuals} : \name{user} \to \name{user} \to \tbool\\
\name{follows} \<x \<y = (\const{john} == x \tand \const{mary} == y)
\\
\phantom{\name{follows} \<x \<y =}
\negphantom{{}\tor}
\tor (\const{mary} == x \tand \const{john} == y)
\\
\phantom{\name{follows} \<x \<y =}
\negphantom{{}\tor}
\tor \ldots
\\
\name{mutuals} \<x \<y = \name{follows} \<x \<y \mathrel{\mathbf{and}} \name{follows} \<y \<x
\end{code}

%% \noindent
%% Unfortunately, this fails to capture the desired behavior of the logic program in two ways:

%% \begin{description}
%% \item[Performance:] Every call to \name{mutuals} does two linear searches by calling \name{follows}, where Datalog would do a quasilinear relational join \emph{once} and re-use its result as a look up table.%
%%   %\footnote{We could solve this by memoizing \name{follows} and \name{mutuals}. Indeed, memoization foreshadows our approach, in that it (dynamically and lazily) turns behavior into data. However, memoization alone does not support enumeration, existential quantification, or aggregation.}

%% \item[Semantics:] Functions in functional and imperative languages are directional: they take inputs to outputs.
%%   Thus $\name{mutuals} \<x \<y$ checks whether specific users $x,y$ are mutuals; but we cannot use it to enumerate mutual follows, nor to find the mutuals $y$ of a given user $x$.
%% \end{description}

\newcommand\emdash{\unskip\ ---\ \ignorespaces}
%\renewcommand\emdash{\unskip\thinspace---\thinspace\ignorespaces}
%\renewcommand\emdash{\unskip\ --\ \ignorespaces}

\noindent
However, functions in functional and imperative languages are unidirectional: they take inputs to outputs.
Evaluating $\name{mutuals} \<x \<y$ will \emph{test} if fixed users $x, y$ are mutuals \emdash but unlike in logic programming, we cannot use it to \emph{enumerate} mutual follows, e.g.\ find the mutuals $y$ of a fixed user $x$.%
%% \footnotemark{}

%% \footnotetext{%
%%   There is also the matter of efficiency.
%%   Each call to $\name{mutuals} \<x \<y$ in a functional language performs two linear scans through \name{follows}.
%%   Bottom-up logic programming languages like Datalog pre-compute tables for \texttt{follows} and \texttt{mutuals}, allowing logarithmic look-up.
%%   However, top-down logic programming languages like Prolog and miniKanren use backtracking search, behaving similarly to our functional translation on this example; and memoization (\emph{tabling} in top-down logic programming) can amortize these costs over repeated tests.
%%   Memoization alone, however, cannot handle existential quantification or aggregation, as we shall see.
%% }

For the same reason, direct functional translation cannot handle \emph{existential quantification} (or in database parlance, \emph{projections}). Consider this logic program to find directors and actors who have worked on the same film:

\begin{lstlisting}
directed_actor(Director, Actor) :-
  directed_film(Director, Film),
  starred_in(Film, Actor).
\end{lstlisting}

\noindent
Since \texttt{Film} does not appear in the head of the rule, the head is derivable if there \emph{exists} a \texttt{Film} satisfying the body.
To embed this functionally, we could assume a function $\name{exists} : \fa{\alpha} (\alpha \to \tbool) \to \tbool$:

\begin{code}
%% \name{directedFilm} : \name{person} \to \name{film} \to \tbool = \dots\\
%% \name{starredIn} : \name{film} \to \name{person} \to \tbool = \dots\\
\name{directedActor} : \name{person} \to \name{person} \to \tbool\\
\name{directedActor} \<\name{director} \<\name{actor} =\\
\quad \name{exists}\<(\fn{\name{film}} \name{directedFilm} \<\name{director} \<\name{film}
\\
\quad \phantom{\name{exists}\<(\fn{\name{film}}}
\negphantom{{}\tand{}}
\tand \name{starredIn} \<\name{film} \<\name{actor})
\end{code}

\noindent
Yet how can we implement $\name{exists} \< f$ for an arbitrary type $\alpha$ and an arbitrary function $f : \alpha \to \tbool$?
We are stymied by unidirectionality: all we can do with $f$ is give it inputs; since the input type is arbitrary, we cannot generate any.%
\footnote{
  Of course, for specific types or functions we may be able to make headway.
  One can define \name{exists} for any finite type $\alpha$ by enumerate-and-test.
  %% For arbitrary $f$, one might expect this is the best you can do:
  As determining $\name{exists}\<f = \name{false}$ seems to require applying $f$ to every possible input, one might expect the infinite cases to be impossible.
  But surprisingly, \name{exists} can be implemented for some infinite types by exploiting continuity \citep{seemingly-impossible-functional-programs}.
}

Logic programming supports existential quantification by rejecting input $\rightarrow$ output directionality:
much as a function is a set of input-output tuples, a predicate can enumerate its tuples \emdash or rather, those whose output is true (any non-enumerated inputs are implicitly false).
%
We call these inputs the \emph{support} of a predicate.
%
If we can enumerate $f$'s support, calculating $\name{exists}\<f$ is simple: is the support non-empty?

This suggests we can make the direct functional translation of logic programs behave correctly if we equip our functions with \emph{enumerable support.}
%
%% "To simplify the problem" isn't enough motivation for finiteness!
\todo{To simplify the problem,} we focus here on \emph{finite} support, and sketch only a fragment of a type system for ensuring finite support, postponing metatheory and implementation.
%
Intriguingly, the typing rules for ensuring finite support resemble, though inexactly, type systems for graded effects, as well as making use of \emph{relevance} typing (a substructural system where each variable must be used \emph{at least once,} a loosening of linear typing's \emph{exactly once}).

%% \footnotetext{
%%   \emph{Productive} meaning that each enumeration step terminates; in particular we must be able to decide whether the support is empty.
%% }

%% Additionally, the functional style makes it natural to add two powerful features: (1) higher order computation, by allowing both finitely supported functions (represented as \emph{data,} e.g.\ a table) and arbitrary, possibly higher-order functions (represented as \emph{code}); and (2) finitely supported functions need not be boolean; finite support is definable for any \emph{pointed} type, that is, a type with a designated ``zero'' value \todo{TODO finish}

%% \noindent
%% \todo{TODO: point out that this is similar to matrix multiplication}
%% \[
%% (A \cdot B)_{ik} = \sum_j A_{ij} \cdot B_{jk}
%% \]

%% \begin{lstlisting}
%% reach(start).
%% reach(Y) :- reach(X), edge(X,Y).
%% \end{lstlisting}


%% ---------- TYPING RULES ----------
\begin{figure*}

  \[\begin{array}{rccl}
    \text{types} & A,B &\bnfeq&
    P
    \bnfor A \cto B
    \bnfor A \times B
    %\bnfor 0
    \bnfor \dots
    \\
    \text{pointed types} & P,Q &\bnfeq&
    P \with Q \bnfor P \otimes Q
    \bnfor P \lto Q
    \bnfor A \fto P
    %% \bnfor \tmaybe A
    \bnfor \tbool
    \\
    \text{terms} & e &\bnfeq&
    x
    \bnfor t
    \bnfor \fn{x} e \bnfor e_1\<e_2
    \bnfor (e_1,\, e_2) \bnfor \pi_i \<e
    %\bnfor \tabsurd e
    \bnfor \dots
    \\
    \text{pointed terms} & t,u &\bnfeq&
    x
    \bnfor \tnil
    \bnfor \fn{x} t \bnfor t \< u \bnfor t \< x
    \bnfor (t,u) \bnfor \pi_i \<t
    \bnfor \letpair{x}{y}{t}{u}
    \bnfor \leteq x t u
    %\bnfor t \tand u \bnfor t \tor u
    %\bnfor \tjust e \bnfor \letjust x t u
    \\
    \text{contexts} & \G,\O &\bnfeq& \emptycx \bnfor \G, \hyp x A
    \\
    \text{pointed contexts} & \D%,\L % \Lambda \Psi
    &\bnfeq& \emptycx \bnfor \D, \hyp x P
    \\
    %% \text{finite support contexts} & \O &\bnfeq& \emptycx \bnfor \O, \hyp x A
    %% \\
  \end{array}\]

  \begin{mathpar}
    \infer*[right=include]{\J t P \G \emptycx \emptycx}{\cJ t P \G}

    \infer*[right=var]{~}{\J x P \G {\hyp x P} {\emptycx}}

    \infer*[right=nil]{~}{\J \tnil P \G \D \O}

    %% FINITE & LINEAR MAPS
    \infer*[right=$\lto$\,i]{
      \J t Q \G {\D, \hyp x P} {\emptycx}
    }{
      \J {\fn x t} {P \lto Q} \G \D {\emptycx}
    }

    \infer*[right=$\fto$\,i]{
      \J t P \G \D {\O, \hyp x A}
    }{
      \J {\fn x t} {A \fto P} \G \D \O
    }

    \infer*[right=$\lto$\,e]{
      \J t {P \lto Q} \G {\D_1} {\O_1}
      \\\\
      \J u {P} {\G,\O_1} {\D_2} {\O_2}
    }{
      \J{t\<u}{Q}{\G}{\D_1 \cup \D_2}{\O_1, \O_2}
    }

    \infer*[right=$\fto$\,e]{
      \J t {A \fto P} \G {\D} {\O}
    }{
      \J {t\<x} P \G {\D} {\O, x : A}
    }

    \infer*[right=$\with$\,i]{
      \J t P \G \D \O
      \\\\
      \J u Q \G \D \O
    }{
      \J {(t,u)} {P \with Q} \G \D \O
    }

    \infer*[right=$\ox$\,i]{
      \J t P \G {\D_1} {\O_1}
      \\\\
      \J u Q {\G, \O_1} {\D_2} {\O_2}
    }{
      \J{(t, u)}{P \ox Q}{\G}{\D_1 \cup \D_2}{\O_1, \O_2}
    }

    \infer*[right=$\with$\,e]{
      \J t {P_1 \with P_2} \G \D \O
    }{
      \J {\pi_i\<t} {P_i} \G \D \O
    }

    \infer*[right=$\ox$\,e]{
      \J t {P \ox Q} \G {\D_1} {\O_1}
      \\\\
      \J s Q {\G, \O_1} {\D_2, \hyp{x}{P}, \hyp{y}{Q}} {\O_2}
    }{
      \J{\letpair{x}{y} t s}{Q}{\G}{\D_1 \cup \D_2}{\O_1, \O_2}
    }

    \infer*[right=let]{
      \J t P \G {\D_1} {\O_1}
      \\\\
      \J u Q {\G, \O_1} {\D_2, x : P} {\O_2}
    }{
      \J {\leteq x t u} Q \G {\D_1 \cup \D_2} {\O_1,\O_2}
    }
  \end{mathpar}

  \caption{Sketch of syntax and type system}
  \label{fig-system}
\end{figure*}


\section{Type system}

\newcommand\defeq\triangleq
%\renewcommand\defeq{\overset{\text{def}}{=}}

\Cref{fig-system} gives a sketch of the syntax and typing rules for a simply-typed $\lambda$-calculus capturing finite support, and \cref{fig-primitives} gives types for primitive functions on booleans.
We consider two kinds of types, normal $A,B$ and pointed $P,Q$.
A pointed type $P$ is a type with a designated point, $\tnil_P \in P$.
A function $f : A \fto P$ is finitely supported iff $\name{support}\<f \defeq \{x \in A : f\<x \ne \tnil_P\}$ is finite.
This generalizes (finite) support to non-boolean functions.

\begin{figure}
  \[\begin{array}{r@{\hskip 0.3em}l}
  %\begin{align*}
    \name{true} &: \tbool\\
    \name{and} &: \tbool \ox \tbool \lto \tbool\\
    \name{or} &: \tbool \with \tbool \lto \tbool\\
    \name{exists} &: (A \fto \tbool) \lto \tbool\\
    \name{==} &: A \cto (A \fto \tbool)
  %\end{align*}\vspace{-1.5\baselineskip}
  \end{array}\]\vspace{-1\baselineskip}
  \caption{Primitive functions}
  \label{fig-primitives}
\end{figure}

We write $f : A \fto P$ for finitely supported maps, besides which we have two more kinds of map: unrestricted $g : A \cto B$ and point-preserving $h : P \lto Q$, meaning that $h(\tnil_P) = \tnil_Q$.
%
Point-preserving maps are useful because they preserve finite support: if $h$ preserves points then $\fn x h \<(f \<x)$ has finite support if $f$ does.

We have two typing judgments, which we can gloss using our function types:

\begin{center}
  \begin{tabular}{l@{\hskip 2em}l}
    \em Judgment & \em Rough interpretation
    \\
    $\cJ e A \G$ & $\G \cto A$
    \\
    $\J t P \G \D \O$ & $\G \cto (\D \lto (\O \fto P))$
  \end{tabular}
  %% \begin{tabular}{r@{\hskip 1.5em}l@{\hskip 1.5em}l}
  %%   \em Judgment
  %%   & $\cJ e A \G$
  %%   & $\J t P \G \D \O$
  %%   \\
  %%   \em Interpretation
  %%   & $\G \cto A$
  %%   & $\G \cto (\D \lto (\O \fto P))$
  %% \end{tabular}
\end{center}

\noindent
The first, $\cJ e A \G$, types `unrestricted' terms that do not interact with finite support or pointed types; it is defined by the usual rules of simply typed $\lambda$ calculus (with $A \cto B$ as function type), which we omit for brevity, plus \textsc{include}.

In the second, $\J t P \G \D \O$, the three typing contexts correspond to our three kinds of function, resembling linear-nonlinear logic.
%
The unrestricted context $\G$ obeys the usual structural rules.
%
The point-preserving context $\D$ obeys the substructural rules of \emph{relevance logic:} variables must be used \emph{at least} once.
Dropping variables is unsound; for example, constant functions like $\fn x 5$ do not preserve \tnil.
%
Finally, the finitely supported context $\O$ does not even have a variable rule! After all, $\fn x x$ is not finitely supported.
%
Instead, eliminating a finitely supported variable requires applying a finitely supported map (\textsc{$\fto\,$e}).
This resembles a monad graded by the context $\O$; applying a finitely supported map is graded monadic join.

The unrestricted and finitely supported contexts interact via \emph{grounding.} For instance, in $\leteq x t u$, variables finitely supported by $t$ may be used \emph{unrestricted} in $u$ so long as $u$ uses $x$ in a point-preserving way.
This internalizes the fact that point-preserving maps preserve finite support.

Our pointed types have the two products characteristic of linear logic: the additive direct product $P \with Q$ of pairs with $\tnil_{P \with Q} = (\tnil_P, \tnil_Q)$; and the multiplicative tensor product $P \otimes Q$, which is the ``smash'' quotient of the direct product by $\tnil_{P \otimes Q} = (\tnil_P, y) = (x, \tnil_Q)$ for any $x : P$ or $y : Q$.
The easiest example of this distinction is that \name{and} takes $\tbool \ox \tbool$, because $\name{and} \<(x,y)$ preserves points in both arguments: it is false ($= \tnil_\tbool$) if either $x$ or $y$ is. But $\name{or} \<(x,y) = \tnil$ only if both $x,y$ are $\tnil$, so $\name{or}$ accepts $\tbool \with \tbool$.


%% \noindent
%% \todo{NUMEROUS PROBLEMS:}

%% 1. shouldn't $\leteq x t u$ be derivable as syntax sugar? what even is its cut reduction / $\beta$-reduction / substitution-based operational semantics?

%% 2. how do we make \name{or} definable? what about \name{exists}?
%% \name{or} is definable via outer-join property, $\name{outer} : \tmaybe \<A \with \tmaybe B \lto \tmaybe (A + B + A \times B)$? yech! and this doesn't handle exists. could have a ``comm. monoid'' typeclass, $P$ a c.mon if it admits $\name{plus} : P \with P \lto P$ and $\name{sum} : (A \fto P) \lto P$, but this doesn't allow defining \name{outer}, does it?


%% \section{Challenges}

%% \subsection{implementation strategies}

%% when do you materialize?
%% (at duplication, I guess?? hm, another layer of linear nonlinear???!?)

%% \subsection{alternating contexts?}

%% \subsection{recursively defined `finite' maps}

%% Recursion: what should it do? how do we give it semantics, operational or denotational? how do we implement it?

%% productive streaming nontermination a la lambda join?
%% iterate to fixed point and only then yield a la Datalog?
%% some hybrid?

%% \subsection{interaction with normal recursive functions}

%% should we even allow them? should we have a 2-level language where the functional bit is metaprogramming for the first-order finite bit?


%% \section{Lipsum}
%% \lipsum


%% \section{Contributions}

%% Our main contributions are: \todo{TODO}


%% %% Bibliography
%% \bibliographystyle{abbrvnat}
%% \bibliography{datafun}

\end{document}
