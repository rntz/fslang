\documentclass{article}

\usepackage[letterpaper,
  hratio=1:1,vratio=3:4,
  %text={152mm,228mm}, % 2:3
  text={162mm,228mm}, % 5:7
]{geometry}

%% FONTS & TYPOGRAPHY

%% %% ---------- OPTION 1: CHARTER ----------
%% \usepackage[sups,scaled=.96]{XCharter}
%% \usepackage[scaled=1.010417965903645]{inconsolata}
%% \usepackage[scaled=0.95,proportional,semibold]{sourcesanspro}
%% %\usepackage[osf,scaled=1.007,scaled=1.033,scaled=1.02,]{AlegreyaSans}
%% \usepackage{eulervm}
%% \edef\zeu@Scale{0.99}
%% \PassOptionsToPackage{scaled=\zeu@Scale,bbscaled=\zeu@Scale,scrscaled=\zeu@Scale}{mathalfa}
%% \RequirePackage[cal=cm]{mathalfa}
%% %% Make sure \mathbold and \mathvar are defined.
%% \linespread{1.06667}            %eyeballed
%% \usepackage{amsmath,amsthm}
%% \usepackage{amssymb}            %\multimap

%% ---------- OPTION 2: PALATINO ----------
\usepackage[scaled=.96,largesc,nohelv]{newpxtext}
\usepackage[scaled=1.0176]{biolinum}
\usepackage[scaled=0.964203422764601]{inconsolata}
\edef\zeu@Scale{.96}
%% HACK: newpxtext doesn't respect fontaxes' \tbfigures! :( but it _does_
%% define, eg., \tlfstyle! hm...
\usepackage{fontaxes}
\renewcommand\tbfigures{}
\renewcommand\lnfigures{}
%\linespread{1.023435141098595}
\linespread{1.08}              % eyeballed 2025-02-06
%% %% NEWPXMATH %%
%% \usepackage[scaled=.96,vvarbb,smallerops]{newpxmath}
%% \usepackage[bb=ams,bbscaled=.96,cal=cm]{mathalfa}
%% EULER MATH %%
\usepackage{eulervm}
\PassOptionsToPackage{scaled=\zeu@Scale,bbscaled=\zeu@Scale,scrscaled=\zeu@Scale}{mathalfa}
\usepackage[cal=cm]{mathalfa}
\usepackage{amsmath,amsthm}
\usepackage{amssymb}            %\multimap

%% %% END FONTS & TYPOGRAPHY

\usepackage[dvipsnames]{xcolor}
%% cleveref must be loaded AFTER hyperref & amsmath, or it errors; and AFTER
%% amsthm and BEFORE we define theorem styles, otherwise (eg) it calls lemmas
%% "Theorem"s.
\usepackage{hyperref,url}
%\usepackage{amsmath,amsthm}
\usepackage[nameinlink,noabbrev]{cleveref}
\usepackage{mathpartir}
\usepackage{mathtools}          %\Coloneqq, \dblcolon
%\usepackage{amssymb}            %\multimap
\usepackage{stmaryrd}           %\llbracket, \shortrightarrow
\usepackage{array}              %\setlength\extrarowheight
\usepackage{booktabs}           %\midrule
\usepackage{accents}            % for \underaccent

\newcommand\crefrangeconjunction{--} % use en-dashes for ranges.
\creflabelformat{equation}{#2#1#3}
\crefformat{footnote}{#2\footnotemark[#1]#3} % referencing footnotes

\newcommand\ensuretext[1]{{\ifmmode\text{#1}\else{#1}\fi}}
\newcommand\strong[1]{{\bfseries#1}}
\newcommand\interpunct{\ensuretext{\textperiodcentered}}

\newcommand\G\Gamma
\newcommand\D\Delta
\renewcommand\O\Omega
\renewcommand\L\Lambda

\newcommand\lto\multimap        % "linear \to"
%\newcommand\fto{\underset{\textsf{fin}}{\to}}           % finite map
\newcommand\fto{\Rightarrow}
\newcommand\x\times
\newcommand\ox\otimes
\newcommand\tbool{\ensuremath{2}}
\newcommand\tmaybe[1]{\ensuremath{{#1}_{*}}}
\renewcommand\tmaybe[1]{\ensuremath{\text{Maybe}~{#1}}}

\newcommand\undertype[2]{\underaccent{\textsf{\upshape\scshape#1}}{#2}}
\newcommand\eqt[1]{\undertype{eq}{#1}}

\newcommand\bnfeq{\Coloneqq}
\newcommand\bnfor{\mathrel{\,|\,}}

%\newcommand\emptycx\varepsilon
\newcommand\emptycx{{\cdot}}
%\renewcommand\emptycx{\,}
%\renewcommand\emptycx{\text{--}}
%\renewcommand\emptycx{\textunderscore}
\newcommand\hyp[2]{{#1} \of {#2}}
\newcommand\hs{,}             %hypothesis separator

\newcommand\of{\mathbin{:}}
%\renewcommand\of{~}
\newcommand\cxsep{\mathbin{/}}

\newcommand\cJ[3]{{#3} \,\vdash\, {#1} : {#2}}
\newcommand\J[5]{{#3} \cxsep {#4} \cxsep {#5} \,\vdash\, {#1} : {#2}}
%\newcommand\pJ[5]{{#3} \cxsep {#4} \,\vdash\, {#1} : {#2} \dashv {#5}}
%\newcommand\pJ[5]{{#3} \cxsep {#4} \cxsep {#5} \,\vdash\, {#1} : {#2}}
\newcommand\pJ[5]{{#3} \cxsep {#4} \cxsep {#1} : {#2} \,\vdash\, {#5}}
%\renewcommand\pJ[5]{{#3} \cxsep {#4} \,\vdash\, {#1} : {#2} \rightsquigarrow {#5}}
%% \renewcommand\pJ[5]{{#3} \cxsep {#4} \cxsep {#2} \,\vdash\, {#1} : {#5}}

\newcommand\ppJ[6]{{#3} \cxsep {#4} \cxsep {#1} : {#2} \,\vdash\, {#5} \cxsep {#6}}
%\renewcommand\ppJ[6]{{#3} \cxsep {#4} \,\vdash\, {#1} : {#2} \rightsquigarrow {#5} \cxsep {#6}}

%% %% Hide terms and hypotheses.
%% \renewcommand\J[5]{{#3} \cxsep {#4} \cxsep {#5} \,\vdash\, {#2}}
%% \renewcommand\hyp[2]{#2}

\newcommand\fname\textit        %function name
\newcommand\cname\text          %constructor name

\newcommand\fnspace\:           %binder space
\newcommand\fn[1]{\lambda{#1}.\fnspace}
\newcommand\<\;                 %function application space

\newcommand\cnil{\cname{nil}}
\newcommand\cjust{\cname{just}}
\newcommand\ctrue{\cname{true}}
\newcommand\cfalse{\cname{false}}

\newcommand\tnil{\cnil}
\newcommand\tjust[1]{\cjust\<{#1}}
\newcommand\letin[2]{\text{let}~{#1}~\text{in}~{#2}}
\newcommand\leteq[3]{\letin{{#1} = {#2}}{#3}}
\newcommand\letjust[3]{\leteq{\tjust {#1}}{#2}{#3}}
\newcommand\letpair[4]{\leteq{({#1}, {#2})}{#3}{#4}}
\newcommand\tand{\mathrel{\textsf{and}}}
\newcommand\tor{\mathrel{\textsf{or}}}
\newcommand\ttrue{\ctrue}
\newcommand\tfalse{\cfalse}
\newcommand\tabsurd{\fname{absurd}\<}

\newcommand\peq{{!}}



\newcommand\isa{\mathrel{\,:\,}}
\newcommand\den[1]{\llbracket{#1}\rrbracket}

\newcommand\todocolor{\color{red}}
\newcommand\todo[1]{\ensuretext{\todocolor #1}}
\newcommand\xxx{\todo{XXX}}

\setlength\parindent{0pt}
\setlength\parskip{2ex}

\begin{document}


\noindent\textsc{quick guide to notation}

\newcommand\support{\ensuremath{\fname{support}}}
\newcommand\supportof[1]{\ensuremath{\support~{#1}}}
\newcommand\defeq{\triangleq}

\begin{tabular}{llp{10cm}}%{rp{25mm}p{110mm}}
  \it Syntax & \it Name & \it Interpretation\\
  $A,B$ & Types & Sets\\
  $A \to B$ & Functions & Functions\\
  $P,Q$ & Pointed types & Sets with a designated `point' or `zero', written $\tnil$, or $\tnil_P$ to make the type explicit.
  For instance, booleans with $\tnil = \text{false}$, or the integers with $\tnil = 0$.
  \\
  $P \lto Q$ & Point-preserving maps & Functions that preserve $\tnil$, that is, $f(\tnil_P) = \tnil_Q$.
  \\
  $A \fto P$ & Finite maps &
  Functions $f : A \to P$ such that $\supportof{f} \defeq \{ x : f(x) \ne \tnil \}$ is finite.
  We represent these in memory as \emph{input-output tables}, listing only pairs $f(x) = y$ where $y \ne \tnil$; finite support means there are finitely many such pairs.
\end{tabular}


\noindent
\textsc{grammars}

\[
\begin{array}{rccl}
  \text{types} & A,B &\bnfeq&
  P \bnfor 0 \bnfor A \times B \bnfor A \to B
  % \bnfor A + B
  \\
  \text{pointed types} & P,Q &\bnfeq&
  \tmaybe A \bnfor P \times Q \bnfor P \otimes Q
  %\bnfor P \oplus Q
  \bnfor P \lto Q
  \bnfor A \fto P
  \\
  \text{terms} & e &\bnfeq& x \bnfor \fn{x} e \bnfor e_1\<e_2
  \bnfor (e_1,\, e_2) \bnfor \pi_i \<e \bnfor \tabsurd e \bnfor t
  \\
  \text{pointed terms} & t,u &\bnfeq&
  x \bnfor \fn{x} t \bnfor t \< u \bnfor t \< p
  \bnfor (t,u) \bnfor \pi_i \<t \bnfor \letpair{x}{y}{t}{u}
  \\
  &&&
  \tnil \bnfor \tjust e
  \bnfor \letjust x t u
  \\
  &&& \leteq x t u
  \\
  \text{patterns} & a,b &\bnfeq&
  x
  \bnfor p
  \bnfor \peq {\todocolor e}   %should this be a term or a pointed term?
  \bnfor (a, b)
  \\
  \text{pointed patterns} & p,q &\bnfeq&
  x \bnfor \tjust a \bnfor (p,q)
  \\
  \text{contexts} & \G,\O &\bnfeq& \emptycx \bnfor \G, \hyp x A
  \\
  \text{pointed contexts} & \D,\L % \Lambda \Psi
  &\bnfeq& \emptycx \bnfor \D, \hyp x P
  \\
  %% \text{finite support contexts} & \O &\bnfeq& \emptycx \bnfor \O, \hyp x A
  %% \\
\end{array}
\]


\noindent\textsc{judgments and their rough interpretation}

\begin{tabular}{l@{\hskip 2em}l@{\hskip 2em}l@{\hskip 2em}l}
  \em Context & \em Example & \em Interpretation & \em Empty context interpretation\\
  $\G,\O$
  & $\hyp{x_1}{A_1},\, \dots,\, \hyp{x_n}{A_n}$
  & $A_1 \times \dots \times A_n$
  & $1 = \tmaybe{0}$
  \\
  $\D,\L$
  & $\hyp{x_1}{P_1},\ \, \dots,\, \hyp{x_n}{P_n}$
  & $P_1 \ox \dots \ox P_n$
  & $2 = \tmaybe{1} = \tmaybe{(\tmaybe{0})}$
\end{tabular}

\begin{tabular}{l@{\hskip 2em}l@{\hskip 2em}l}
  \em Judgment & \em Interpretation
  \\
  $\cJ e A \G$ & $\G \to A$
  \\
  $\J t P \G \D \O$ & $\G \to \D \lto \O \fto P$
  \\
  $\pJ a A \G \D \O$
  & $\G \to \D \lto A \to \tmaybe{\O}$
  \\
  $\ppJ p P \G \D \O \L$
  & $\G \to \D \lto P \lto (\tmaybe{\O}) \ox \L$
  %% & \textit{note the swapping of $A, \O$!}
\end{tabular}

%% \[
%% \begin{array}{l@{\hskip 2em}l@{\hskip 2em}l@{\hskip 2em}l}
%%   \textit{Context} & \textit{Example} & \textit{Interpretation}
%%   & \textit{Empty context}
%%   \\
%%   \G,\O & \hyp{x_1}{A_1},\, \dots,\, \hyp{x_n}{A_n} & A_1 \times \dots \times A_n
%%   & 1 = \tmaybe{0}
%%   \\
%%   \D    & \hyp{x_1}{P_1},\, \dots,\, \hyp{x_n}{P_n} & P_1 \ox \dots \ox P_n
%%   & 2 = \tmaybe{1} = \tmaybe{(\tmaybe{0})}\\
%%   \\
%%   \\
%%   \textit{Judgment} & \textit{Interpretation}
%%   \\
%%   \cJ e A \G
%%   & \G \to A
%%   \\
%%   \J t P \G \D \O
%%   & \G \to \D \lto \O \fto P
%%   \\
%%   \pJ a A \G \D \O
%%   & \G \to \D \lto A \to \tmaybe{\O}
%%   & \multicolumn{2}{l}{\todo{why not $\G \to A \to \D \lto \tmaybe{\O}$?}}
%%   \\
%%   \pJ p P \G \D \O
%%   & \G \to \D \lto P \lto \O
%%   %% & \textit{note the swapping of $A, \O$!}
%% \end{array}
%% \]

\noindent
\todo{Maybe this should be redone to allow arbitrary nesting/interleaving of all three kinds of context. See ``PROBLEMS'', below. (Aren't finitely supported maps a graded monad over pointed sets? Or not?)}


\noindent\textsc{inference rules}

\emph{Term typing, }$\cJ e A \G$
\begin{mathpar}
  \infer*[right=var]{\hyp x A \in \G}{\cJ x A \G}

  \infer*[right=include]{\J t P \G \emptycx \emptycx}{\cJ t P \G}

  \infer*[right=absurd]{\cJ e 0 \G}{\cJ {\tabsurd e} A \G}
  \\
  \text{and the other obvious rules...}
\end{mathpar}


\emph{Pointed term typing, }$\J t P \G \D \O$
\begin{mathpar}
  %% UNIVERSAL RULES
  \infer*[right=var]{~}{\J x P \G {\hyp x P} {\emptycx}}

  \infer*[right=include]{\cJ e P \G}{\J e P \G \emptycx \emptycx}

  \infer*[right=nil]{~}{\J {\tnil} P \G \D \O}

  \infer*[right=let]{
    \J t P \G {\D_1} {\O_1}
    \\
    \J u Q {\G,\O_1} {\D_2} {\O_2}
  }{
    \J {\leteq x t u} Q \G {\D_1\cup\D_2} {\O_1,\O_2}
  }
  \\
  %% PRODUCTS
  \infer*[right=$\x$\,i]{
    \J t P \G \D \O
    \\
    \J u Q \G \D \O
  }{
    \J {(t,u)} {P \x Q} \G \D \O
  }

  \infer*[right=$\x$\,e]{
    \J t {P_1 \x P_2} \G \D \O
  }{
    \J {\pi_i\<t} {P_i} \G \D \O
  }
  \\
  %% TENSOR PRODUCTS
  \infer*[right=$\ox$\,i]{
    \J t P \G {\D_1} {\O_1}
    \\
    \J s Q {\G\hs \O_1} {\D_2} {\O_2}
  }{
    \J{(t,s)}{P \ox Q}{\G}{\D_1 \cup \D_2}{\O_1\hs \O_2}
  }

  \infer*[right=$\ox$\,e]{
    \J t {P \ox Q} \G {\D_1} {\O_1}
    \\\\
    \J s Q {\G\hs \O_1} {\D_2\hs \hyp{x}{P}\hs \hyp{y}{Q}} {\O_2}
  }{
    \J{\letpair{x}{y} t s}{Q}{\G}{\D_1 \cup \D_2}{\O_1\hs \O_2}
  }
  \\
  %% LINEAR MAPS
  \infer*[right=$\lto$\,i]{
    \J t Q \G {\D, \hyp x P} {\emptycx}
  }{
    \J {\fn x t} {P \lto Q} \G \D {\emptycx}
  }

  \infer*[right=$\lto$\,e]{
    \J t {P \lto Q} \G {\D_1} {\O_1}
    \\
    \J u {P} {\G,\O_1} {\D_2} {\O_2}
  }{
    \J{t\<u}{Q}{\G}{\D_1 \cup \D_2}{\O_1, \O_2}
  }
  \\
  %% MAYBE
  \infer*[right=maybe\,i]{
    \cJ e A \G
  }{
    \J {\tjust e} {\tmaybe A} \G \emptycx \emptycx
  }

  \infer*[right=maybe\,e]{
    \J t {\tmaybe A} \G {\D_1} {\O_1}
    \\
    \J u P {\G, \O_1, \hyp x A} {\D_2} {\O_2}
  }{
    \J {\letjust x t u} {P} \G {\D_1 \cup \D_2} {\O_1,\O_2}
  }
  \\
  %% FINITE MAPS
  \infer*[right=$\fto$\,i]{
    \J t P \G \D {\O, \hyp x A}
  }{
    \J {\fn x t} {A \fto P} \G \D \O
  }

  \infer*[right=$\fto$\,e]{
    \J t {A \fto P} \G {\D_1} {\O_1}
    \\
    \pJ p A {\G,\O_1} {\D_2} {\O_2}
  }{
    \J {t\<p} P \G {\D_1\cup\D_2} {\O_1,\O_2}
  }
\end{mathpar}


\emph{Pattern typing, }$\pJ a A \G \D \O$
\begin{mathpar}
  %% \infer*[right=include \todo{\em - not sound!! let $P = Q_1 \lto Q_2$ and $p = x$}]{
  %%   \ppJ p P \G \D \O \L
  %% }{
  %%   \pJ p P \G \D {\O,\L}
  %% }
  \infer*[right=limited include]{
    \ppJ p P \G \D \O \emptycx
  }{
    \pJ p P \G \D {\O}
  }

  \infer*[right=var]{~}{
    \pJ x A \G \emptycx {\hyp x A}
  }

  \infer*[right=$\x$]{
    \pJ a A \G {\D_1} {\O_1}
    \\
    \pJ b B {\G,\O_1} {\D_2} {\O_2}
  }{
    \pJ {(a,b)} {A \x B} \G \D {\O_1,\O_2}
  }

  \infer*[right=eq]{
    \cJ e {\eqt A} \G
  }{
    \pJ {\peq e} {\eqt A} \G \emptycx \emptycx
  }
\end{mathpar}

The main motivation for \textsc{limited include} is matching on `$\cjust$' in a finite map application:
\begin{mathpar}
  \infer*[right=$\fto$\,e]{
    \infer*[right=var]{~}{
      \J f {\tmaybe{A} \fto P} \G {\hyp f {\tmaybe{A} \fto P}} \emptycx
    }
    \\
    \infer*{
      \infer*[right=just]{
        \infer*[right=var]{~}{
          \pJ x A \G \emptycx {\hyp x A}
        }
      }{
        \ppJ {\tjust x} {\tmaybe A} \G \emptycx {\hyp x A} \emptycx
      }
    }{
      \pJ {\tjust x} {\tmaybe A} \G \emptycx {\hyp x A}
    }
  }{
    \J {f\<(\tjust x)} P \G {\hyp f {\tmaybe{A} \fto P}} {\hyp x A}
  }
\end{mathpar}

However, \textsc{limited include} has several issues:
\begin{enumerate}
\item We need $\D = \emptycx$ in \textsc{limited include} because $\D$ gets tensored together while $\O$ gets cross-producted, and we can't convert $\D_\ox$ into $\D_\x$ in general.

\item We must decide whether a cartesian pattern matches or fails, but we \emph{don't} need this for pointed patterns; this amounts to deciding ``equality with the point'' on the output of pointed pattern matching. This makes me nervous about the semantics of \textsc{limited include}. \todo{Do I have any problematic examples?}

\item Finally, including pointed patterns in patterns causes syntactic ambiguity: does the pattern $x$ signify a cartesian variable or the inclusion of a pointed variable?
\end{enumerate}

\emph{Pointed pattern typing, }$\ppJ p P \G \D \O \L$
\begin{mathpar}
  \infer*[right=var]{~}{
    \ppJ x P \G \D {\emptycx} {\hyp x P}
  }

  \infer*[right=just]{
    \pJ a A \G \D \O
  }{
    \ppJ {\tjust a} {\tmaybe A} \G \D \O {\emptycx}
  }

  \infer*[right=$\ox$]{
    \ppJ p P \G {\D_1} {\O_1} {\L_1}\\
    \ppJ q Q \G {\D_2} {\O_2} {\L_2}
  }{
    \ppJ {(p,q)} {P \ox Q} \G {\D_1 \cup \D_2} {\O_1,\O_2} {\L_1, \L_2}
  }
\end{mathpar}


\noindent\textsc{rough denotational justifications for selected rules}

%% denotational function
\newcommand\dmap[3]{{#1} \to {#2} \lto {#3} \fto}
\newcommand\darg[3]{{#1}\ {#2}\ {#3}}
\newcommand\dfn[3]{\darg{#1}{#2}{#3} \mapsto}
\newcommand\g\gamma
\renewcommand\d\delta
\renewcommand\o\omega

\begin{align*}
  %% Linear variables
  &
  \infer*[right=var]{~}{
    \J x P \G {\hyp x P} {\emptycx}
  }
  &&
  %% \infer{~}{\dmap \G P 1 P}
  P \cong 1 \fto P
  %% &
  %% \darg{\g} p {()} &\mapsto p
  \\[1em]&
  %% A -o B intro
  \infer*[right=$\lto$\,i]{
    \J t Q \G {\D, \hyp x P} {\emptycx}
  }{
    \J{\fn x t}{P \lto Q}{\G}{\D}{\emptycx}
  }
  &&
  %% TODO: Do we need $\Omega=\emptycx$?
  %% Why? Explain this somewhere
  \infer{
    P \lto (1 \fto Q)
  }{
    1 \fto {(P \lto Q)}
  }
  ~\text{but not}~
  \infer{
    P \lto (A \fto Q)
  }{
    A \fto {(P \lto Q)}
  }
  %% &
  %% \darg{\g}{\d}{()} &\mapsto \fn{p} \den t\<\g\<(\d, p)\<()
  \\[1em]&
  %% A -o B elim
  \infer*[right=$\lto$\,e]{
    \J t {P \lto Q} \G {\D_1} {\O_1}
    \\\\
    \J u {P} {\G,\O_1} {\D_2} {\O_2}
  }{
    \J{t\<u}{Q}{\G}{\D_1 \cup \D_2}{\O_1, \O_2}
  }
  &&
  %% TODO: note the implicit use of duplication here!
  \infer{
    \D_1 \lto \O_1 \fto (P \lto Q)
    \\\\
    \O_1 \to \D_2 \lto \O_2 \fto P
  }{
    \D_1 \ox \D_2 \lto \O_1 \x \O_2 \fto Q
  }
  \quad
  \todo{check this}
  %% &
  %% \darg{\g}{(\d_1,\d_2)}{(\o_1,\o_2)} &\mapsto
  %% \den t \<\g \<\d_1 \<\o_1 \<(\den s \<(\g,\o_1) \<\d_2 \<\o_2)
\end{align*}


\[
\setlength{\arraycolsep}{1em}
\def\arraystretch{4}
\begin{array}{ccl}
  \infer[var]{~}{
    \J x P \G {\hyp x P} {\emptycx}
  }

  &
  \infer{~}{\dmap \G P 1 P}

  &
  \dfn{\g}{p}{()} p
  \\
  \infer[$\lto$ intro]{
    \J t Q \G {\D, \hyp x P} {\emptycx}
  }{
    \J{\fn x t}{P \lto Q}{\G}{\D}{\emptycx}
  }

  &  
  \infer{
    \dmap \G {\D \ox P} 1 Q
  }{
    \dmap \G \D 1 {P \lto Q}
  }

  &
  \dfn{\g}{\d}{()} \fn{p} t\<\g\<(\d \ox p)\<()

  %% &\raggedright\sffamily\small
  %% Does this need $\Omega=\emptycx$?
  \\
  \infer[$\lto$ elim]{
    \J t {P \lto Q} \G {\D_1} {\O_1}
    \\\\
    \J t {P} {\G,\O_1} {\D_2} {\O_2}
  }{
    \J{t\<s}{Q}{\G}{\D_1 \cup \D_2}{\O_1, \O_2}
  }

  &
  \infer{
    \G \to \D_1 \to \O_1 \fto P \lto Q
    \\\\
    \G \x \O_1 \to \D_2 \lto \O_2 \fto P
  }{
    \G \to \D_1 \ox \D_2 \lto \O_1 \x \O_2 \fto Q
  }

  &
  \xxx
\end{array}
\]


\noindent\textsc{the structural problem: finite maps don't commute with functions}

Although we haven't included it in our grammar, $A \to P$ is a pointed type, with $\tnil_{A \to P} = \fn x \tnil_P$.
%
Moreover, function spaces commute with point-preserving function spaces: $A \to (P \lto Q) \cong P \lto (A \to Q)$.
%
So we need not worry about the `relative order' of hypotheses in $\G$ and $\D$.
%
We can combine them into one context, or keep them separate, as we please.

Unfortunately, functions and finitely supported maps do \emph{not} commute: $A \to (B \fto P) \ncong B \fto (A \to P)$. As a counterexample, consider equality:

\begin{align*}
\fname{eq} &: A \to A \fto \tbool
&
\fname{eq}' &\!\!\not\,\,: A \fto A \to \tbool
\qquad\color{red}\textit{(ill-typed!)}
\\
\fname{eq}\<x \<y &= 
\begin{cases}
  \ttrue & \text{if }x = y\\
  \tfalse & \text{otherwise}
\end{cases}
&
\fname{eq}'\<y \<x &= 
\begin{cases}
  \ttrue & \text{if }x = y\\
  \tfalse & \text{otherwise}
\end{cases}
\end{align*}

Since $\tnil_\tbool = \tfalse$, for any $x : A$, there are finitely many $y : A$ (exactly one, in fact) such that $\fname{eq} \<x\<y \ne \tnil$. However, $\tnil_{A \to \tbool} = \fn x \tfalse$, so unless $A$ is finite there are infinitely many $y : A$ (all of them, in fact) such that $\fname{eq}'\<y \ne \tnil$.

For the same reason, point-preserving and finite maps do not commute: $P \lto (A \fto Q) \ncong A \fto (P \lto Q)$. It is slightly more inconvenient to construct a counterexample here, but not especially difficult. \todo{Is there a direction of this non-isomorphism that does hold? ie. a map from one side to the other?}

The consequence of this is \todo{that the rules for $\to$ and $\lto$ introduction have to have empty $\O$ contexts.}


\noindent
\todo{\textsc{problems \& todos}}
\begin{enumerate}
\item Can't define $\fname{or} : \tbool \x \tbool \lto \tbool$ within the language! I'm not sure how much of an issue this is. For instance, $\fname{exists} : (A \fto \tbool) \lto \tbool$ almost certainly needs to be primitive. And any commutative monoid $P$ where the point is the monoid's identity element should have methods $(\oplus) : P \times P \lto P$ and $\fname{sum} : (A \fto P) \lto P$; indeed, ``commutative monoid'' is essentially a typeclass with these two methods.

  (Q: In general, does $P \times P \lto P$ give rise to $(A \fto P) \lto P$ and/or vice-versa? A: There's no guarantee $P \times P \lto P$ is commutative. However, $(A \fto P) \lto P$ should give rise to $P \x P \lto P$ because $P \x P \lto (1 + 1) \fto P$.)

\item Can't express $\fname{forall} : (A \fto \tbool) \lto (A \to \tbool) \lto \tbool$, for two reasons:
  \begin{enumerate}
  \item $A \to \tbool$ isn't recognized as a pointed type.
    Could `fix' this by adding $P \bnfeq A \to P$, however...
  \item We still can't give proper intro/elim rules for it because it would require putting \emph{set} variables \emph{after} the linear context $\Delta$. Solution: maybe we need an ordered context which can mix set \& pointed variables? Or a sequence of alternating set/pointed/fin.sup contexts?
  \end{enumerate}
\end{enumerate}

\noindent One way to perhaps solve (1-2) is to stop pussyfooting about with pointed types and use commutative monoids instead. But then tensor product becomes unruly (it's no longer a quotient of the direct product); the ``maybe'' type $\tmaybe A$ isn't a commutative monoid; and perhaps other complexities creep in.

\begin{enumerate}
\item Think about what sideways information passing does to query planning.

  Sometimes it's simple, eg $(\fn{x} f\<x \tand x > 17)$ where $f : A \fto \tbool$.

\item Explain why we need $\Omega=\emptycx$ in ${\lto}\,\textsc{i}$.
\item Explain/justify the implicit duplication (turning $\D_1 \cup \D_2 \lto \D_1 \ox \D_2$) in the semantics of ${\lto}\,\textsc{e}$ and ${\ox}\,\textsc{i}$.

\item Why am I working with $\J t P \G \D \O$ as $\G \to \D \lto \O \fto P$ instead of putting the linear arguments after the f.s. ones, $\G \to \O \fto (\D \lto P)$?
  Or do I want both and need an ordered stack of contexts?

\item \strong{More example programs!} Ones which \emph{use} the $\D$ linear context in pattern matching / in combination with sideways information passing.
\end{enumerate}


\end{document}
