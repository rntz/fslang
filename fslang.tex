\documentclass{article}

\usepackage[letterpaper,
  hratio=1:1,vratio=3:4,
  %text={152mm,228mm}, % 2:3
  text={162mm,228mm}, % 5:7
]{geometry}

%% FONTS & TYPOGRAPHY
\usepackage[sups,scaled=.96]{XCharter}
\usepackage[scaled=1.010417965903645]{inconsolata}
\usepackage[osf,scaled=1.007,scaled=1.033,scaled=1.02,]{AlegreyaSans}
\usepackage[euler-digits]{eulervm}
\edef\zeu@Scale{0.99}
\PassOptionsToPackage{scaled=\zeu@Scale,bbscaled=\zeu@Scale,scrscaled=\zeu@Scale}{mathalfa}
\RequirePackage[cal=cm]{mathalfa}
%% Make sure \mathbold and \mathvar are defined.
\providecommand\mathbold\boldsymbol
\providecommand\mathvar\ensuremath
\linespread{1.06667}            %eyeballed
%% END FONTS & TYPOGRAPHY

\usepackage[dvipsnames]{xcolor}
%% cleveref must be loaded AFTER hyperref & amsmath, or it errors; and AFTER
%% amsthm and BEFORE we define theorem styles, otherwise (eg) it calls lemmas
%% "Theorem"s.
\usepackage{hyperref,url,amsmath,amsthm}
\usepackage[nameinlink,noabbrev]{cleveref}
\usepackage{mathpartir}
\usepackage{mathtools}          %\Coloneqq, \dblcolon
\usepackage{amssymb}            %\multimap
\usepackage{stmaryrd}           %\llbracket, \shortrightarrow
\usepackage{array}              %\setlength\extrarowheight

\newcommand\crefrangeconjunction{--} % use en-dashes for ranges.
\creflabelformat{equation}{#2#1#3}
\crefformat{footnote}{#2\footnotemark[#1]#3} % referencing footnotes

\newcommand\ensuretext[1]{{\ifmmode\text{#1}\else{#1}\fi}}
\newcommand\strong[1]{{\bfseries#1}}
\newcommand\interpunct{\ensuretext{\textperiodcentered}}

\newcommand\G\Gamma
\newcommand\D\Delta
\renewcommand\O\Omega
\newcommand\lto\multimap        % "linear \to"
%\newcommand\fto{\underset{\textsf{fin}}{\to}}           % finite map
\newcommand\fto{\Rightarrow}
\newcommand\x\times
\newcommand\ox\otimes
\newcommand\tbool{\ensuremath{2}}
\newcommand\tmaybe[1]{\ensuremath{{#1}_{*}}}

\newcommand\bnfeq{\Coloneqq}
\newcommand\bnfor{\mathrel{\,|\,}}

%\newcommand\emptycx\varepsilon
\newcommand\emptycx{{\cdot}}
\newcommand\hyp[2]{{#1} \of {#2}}
\newcommand\hs{,}             %hypothesis separator

\newcommand\of{\mathbin{:}}
%\renewcommand\of{~}
\newcommand\cxsep{\mathbin{/}}

\newcommand\cJ[3]{{#3} \,\vdash\, {#1} : {#2}}
\newcommand\J[5]{{#3} \cxsep {#4} \cxsep {#5} \,\vdash\, {#1} : {#2}}
%\newcommand\pJ[5]{{#3} \cxsep {#4} \,\vdash\, {#1} : {#2} \dashv {#5}}
\newcommand\pJ[5]{{#3} \cxsep {#4} \cxsep {#5} \,\vdash\, {#1} : {#2}}

%% %% Hide terms and hypotheses.
%% \renewcommand\J[5]{{#3} \cxsep {#4} \cxsep {#5} \,\vdash\, {#2}}
%% \renewcommand\hyp[2]{#2}

\newcommand\fname\text          %function name
\newcommand\cname\text          %constructor name

\newcommand\fnspace\:           %binder space
\newcommand\fn[1]{\lambda{#1}.\fnspace}
\newcommand\<\;                 %function application space


\newcommand\cnil{\cname{nil}}
\newcommand\cjust{\cname{just}}

\newcommand\tnil{\cnil}
\newcommand\tjust[1]{\cjust\<{#1}}
\newcommand\letin[2]{\text{let}~{#1}~\text{in}~{#2}}
\newcommand\leteq[3]{\letin{{#1} = {#2}}{#3}}
\newcommand\letjust[3]{\leteq{\tjust {#1}}{#2}{#3}}
\newcommand\letpair[4]{\leteq{({#1}, {#2})}{#3}{#4}}
\newcommand\tand{\mathrel{\textsf{and}}}
\newcommand\tor{\mathrel{\textsf{or}}}

\newcommand\peq{{!}}



\newcommand\isa{\mathrel{\,:\,}}
\newcommand\den[1]{\llbracket{#1}\rrbracket}

\newcommand\todo[1]{\ensuretext{\color{red}#1}}
\newcommand\xxx{\todo{XXX}}

%% \setlength\parindent{0pt}
%% \setlength\parskip{2ex}

\begin{document}


\noindent
\textsc{grammars}

\[
\begin{array}{rccl}
  \text{types} & A,B &\bnfeq&
  P \bnfor 0 \bnfor A \times B \bnfor A \to B
  % \bnfor A + B
  \\
  \text{pointed types} & P,Q &\bnfeq&
  \tmaybe A \bnfor P \times Q \bnfor P \otimes Q
  %\bnfor P \oplus Q
  \bnfor P \lto Q
  \bnfor A \fto P
  \\
  \text{terms} & e &\bnfeq& x \bnfor \fn{x} e \bnfor e_1\<e_2
  \bnfor (e_1, e_2) \bnfor \pi_i \<e \bnfor t
  \\
  \text{pointed terms} & t,u &\bnfeq&
  x \bnfor \fn{x} t \bnfor t \< u \bnfor t \< p
  \bnfor (t,u) \bnfor \pi_i \<t \bnfor \letpair{x}{y}{t}{u}
  \\
  &&&
  \tnil \bnfor \tjust e
  \bnfor \letjust x t u         %should this be a pattern?
  \\
  \text{support patterns} & p &\bnfeq& x \bnfor \peq e \bnfor (p,q) \bnfor \tjust p \bnfor ...
  \\
  \text{contexts} & \G &\bnfeq& \emptycx \bnfor \G, \hyp x A
  \\
  \text{pointed contexts} & \D &\bnfeq& \emptycx \bnfor \D, \hyp x P
  \\
  \text{finite support contexts} & \O &\bnfeq& \emptycx \bnfor \O, \hyp x A
  \\
\end{array}
\]

\vskip 1em

\noindent
\textsc{judgments}

\[
\begin{array}{l@{\hskip 2em}l@{\hskip 2em}l}
  \cJ e A \G
  & \G \to A
  \\
  \J t P \G \D \O
  & \G \to \D \lto \O \fto P
  \\
  \pJ p A \G \D \O
  & \G \to \D \lto A \to \tmaybe{\O}
  & \textit{note the swapping of $A, \O$!}
\end{array}
\]

\noindent
\todo{Maybe this should be redone to allow arbitrary nesting/interleaving of all three kinds of context. See ``PROBLEMS'', below. (Aren't finitely supported maps a graded monad over pointed sets? Or not?)}

\vskip 1em

\noindent
\textsc{inference rules and their rough denotational justifications}

%% denotational function
\newcommand\dmap[3]{{#1} \to {#2} \lto {#3} \fto}
\newcommand\darg[3]{{#1}\ {#2}\ {#3}}
\newcommand\dfn[3]{\darg{#1}{#2}{#3} \mapsto}
\newcommand\g\gamma
\renewcommand\d\delta
\renewcommand\o\omega

\begin{align*}
  %% Linear variables
  &
  \infer*[right=var]{~}{
    \J x P \G {\hyp x P} {\emptycx}
  }
  &&
  %% \infer{~}{\dmap \G P 1 P}
  P \cong 1 \fto P
  %% &
  %% \darg{\g} p {()} &\mapsto p
  \\[1em]&
  %% A -o B intro
  \infer*[right=$\lto$\,i]{
    \J t Q \G {\D, \hyp x P} {\emptycx}
  }{
    \J{\fn x t}{P \lto Q}{\G}{\D}{\emptycx}
  }
  &&
  %% TODO: Do we need $\Omega=\emptycx$?
  %% Why? Explain this somewhere
  \infer{
    P \lto (1 \fto Q)
  }{
    1 \fto {(P \lto Q)}
  }
  ~\text{but not}~
  \infer{
    P \lto (A \fto Q)
  }{
    A \fto {(P \lto Q)}
  }
  %% &
  %% \darg{\g}{\d}{()} &\mapsto \fn{p} \den t\<\g\<(\d, p)\<()
  \\[1em]&
  %% A -o B elim
  \infer*[right=$\lto$\,e]{
    \J t {P \lto Q} \G {\D_1} {\O_1}
    \\\\
    \J t {P} {\G,\O_1} {\D_2} {\O_2}
  }{
    \J{t\<s}{Q}{\G}{\D_1 \cup \D_2}{\O_1, \O_2}
  }
  &&
  %% TODO: note the implicit use of duplication here!
  \infer{
    \D_1 \lto \O_1 \fto (P \lto Q)
    \\\\
    \O_1 \to \D_2 \lto \O_2 \fto P
  }{
    \D_1 \ox \D_2 \lto \O_1 \x \O_2 \fto Q
  }
  \quad
  \todo{check this}
  %% &
  %% \darg{\g}{(\d_1,\d_2)}{(\o_1,\o_2)} &\mapsto
  %% \den t \<\g \<\d_1 \<\o_1 \<(\den s \<(\g,\o_1) \<\d_2 \<\o_2)
\end{align*}


\noindent
\todo{PROBLEMS:}
\begin{enumerate}
\item Can't define $\fname{or} : \tbool \x \tbool \lto \tbool$ within the language! I'm not sure how much of an issue this is. For instance, $\fname{exists} : (A \fto \tbool) \lto \tbool$ almost certainly needs to be primitive. And any commutative monoid $P$ where the point is the monoid's identity element should have methods $(\oplus) : P \times P \lto P$ and $\fname{sum} : (A \fto P) \lto P$; indeed, ``commutative monoid'' is essentially a typeclass with these two methods.

  (Q: In general, does $P \times P \lto P$ give rise to $(A \fto P) \lto P$ and/or vice-versa? A: There's no guarantee $P \times P \lto P$ is commutative. However, $(A \fto P) \lto P$ should give rise to $P \x P \lto P$ because $P \x P \lto (1 + 1) \fto P$.)

\item Can't express $\fname{forall} : (A \fto \tbool) \lto (A \to \tbool) \lto \tbool$, for two reasons:
  \begin{enumerate}
  \item $A \to \tbool$ isn't recognized as a pointed type.
  \item Even if it were, we can't give proper intro/elim rules for it because it would require putting \emph{set} variables \emph{after} the linear context $\Delta$. Solution: maybe we need an ordered context which can mix set \& pointed variables? Or a sequence of alternating set/pointed contexts?
  \end{enumerate}
\end{enumerate}

\noindent One way to perhaps solve (1-2) is to stop pussyfooting about with pointed types and use commutative monoids instead. But then tensor product becomes unruly (it's no longer a quotient of the direct product); the ``maybe'' type $\tmaybe A$ isn't a commutative monoid; and perhaps other complexities creep in.

\noindent
\todo{TODO:}

\begin{enumerate}
\item Think about what sideways information passing does to query planning.

  Sometimes it's simple, eg $(\fn{x} f\<x \tand x > 17)$ where $f : A \fto \tbool$.

\item Explain why we need $\Omega=\emptycx$ in ${\lto}\,\textsc{i}$.
\item Explain/justify the implicit duplication (turning $\D_1 \cup \D_2 \lto \D_1 \ox \D_2$) in the semantics of ${\lto}\,\textsc{e}$ and ${\ox}\,\textsc{i}$.

\item Why am I working with $\J t P \G \D \O$ as $\G \to \D \lto \O \fto P$ instead of putting the linear arguments after the f.s. ones, $\G \to \O \fto (\D \lto P)$?
  Or do I want both and need an ordered stack of contexts?

\item \strong{More example programs!} Ones which \emph{use} the $\D$ linear context in pattern matching / in combination with sideways information passing.
\end{enumerate}


\[
\setlength{\arraycolsep}{1em}
\def\arraystretch{4}
\begin{array}{ccl}
  \infer[var]{~}{
    \J x P \G {\hyp x P} {\emptycx}
  }

  &
  \infer{~}{\dmap \G P 1 P}

  &
  \dfn{\g}{p}{()} p
  \\
  \infer[$\lto$ intro]{
    \J t Q \G {\D, \hyp x P} {\emptycx}
  }{
    \J{\fn x t}{P \lto Q}{\G}{\D}{\emptycx}
  }

  &  
  \infer{
    \dmap \G {\D \ox P} 1 Q
  }{
    \dmap \G \D 1 {P \lto Q}
  }

  &
  \dfn{\g}{\d}{()} \fn{p} t\<\g\<(\d \ox p)\<()

  %% &\raggedright\sffamily\small
  %% Does this need $\Omega=\emptycx$?
  \\
  \infer[$\lto$ elim]{
    \J t {P \lto Q} \G {\D_1} {\O_1}
    \\\\
    \J t {P} {\G,\O_1} {\D_2} {\O_2}
  }{
    \J{t\<s}{Q}{\G}{\D_1 \cup \D_2}{\O_1, \O_2}
  }

  &
  \infer{
    \G \to \D_1 \to \O_1 \fto P \lto Q
    \\\\
    \G \x \O_1 \to \D_2 \lto \O_2 \fto P
  }{
    \G \to \D_1 \ox \D_2 \lto \O_1 \x \O_2 \fto Q
  }

  &
  wub
\end{array}
\]


\noindent
\todo{wub}
\begin{mathpar}
  \infer[$\lto$ elim]{
    \J t {P \lto Q} \G {\D_1} {\O_1}
    \\
    \J t {P} {\G,\O_1} {\D_2} {\O_2}
  }{
    \J{t\<s}{Q}{\G}{\D_1 \cup \D_2}{\O_1, \O_2}
  }

  \infer{
    f \isa \G \to \D_1 \to \O_1 \fto P \lto Q
    \\
    g \isa \G \x \O_1 \to \D_2 \lto \O_2 \fto P
  }{
    \xxx \isa \G \to \D_1 \ox \D_2 \lto \O_1 \x \O_2 \fto Q
  }

  \text{Note the silent use of duplication.}
  \\
  \infer[$\otimes$ intro]{
    \J t P \G {\D_1} {\O_1}
    \\
    \J s Q {\G\hs \O_1} {\D_2} {\O_2}
  }{
    \J{(t,s)}{P \otimes Q}{\G}{\D_1 \cup \D_2}{\O_1\hs \O_2}
  }

  \infer[$\otimes$ elim]{
    \J t {P_1 \otimes P_2} \G {\D_1} {\O_1}
    \\
    \J s Q {\G\hs \O_1} {\D_2\hs \hyp{x_1}{P_1}\hs \hyp{x_2}{P_2}} {\O_2}
  }{
    \J{\letpair{x_1}{x_2} t s}{Q}{\G}{\D_1 \cup \D_2}{\O_1\hs \O_2}
  }

  \infer[$\fto$ intro]{
    \J t P \G \D {\O, \hyp x A}
  }{
    \J{\fn x t} {A \fto P} \G \D \O
  }

  \infer[$\fto$ elim]{
    \J t {A \fto P} \G {\D_1} {\O_1}
    \\
    \pJ p A {\G,\O_1} {\D_2} {\O_2}
  }{
    \J {t \<p} P \G {\D_1 \cup \D_2} {\O_1, \O_2}
  }
\end{mathpar}

\end{document}
